\chapter{\abstractname}

%TODO: Abstract
In order to remain competitive in the ongoing globalization, companies are forced to optimize their productions and processes. Using modern smart technologies like Internet of Things (IoT) and Machine to machine communication (M2M) advances the digitization and automation in production facilities. The immense amount of generated data offers great opportunities for Predictive Maintenance approaches. By analysing machine and production data, machines can be proactively maintained, thereby increasing quality standards and productivity. Since ball screws are one of the most common parts in industrial machines, their health monitoring is especially interesting. Due to strong abrasion ball screws change a lot during lifetime.
Therefore, a strong shift in the data distribution can be observed between the working conditions. This makes the fault diagnosis problem more challenging. In order to tackle this problem deep-learning based domain adaption approaches, which are common in the Computer Vision community are promising. In this work, the applicability of such approaches to the given predictive maintenance problem is investigated. 




\makeatletter
\ifthenelse{\pdf@strcmp{\languagename}{english}=0}
{\renewcommand{\abstractname}{Kurzfassung}}
{\renewcommand{\abstractname}{Abstract}}
\makeatother

\chapter{\abstractname}

%TODO: Abstract in other language
\begin{otherlanguage}{ngerman} % TODO: select other language, either ngerman or english !
Um in der fortschreitenden Globalisierung wettbewerbsfähig zu bleiben, sind Unternehmen gezwungen, ihre Produktionen und Prozesse zu optimieren. Der Einsatz moderner intelligenter Technologien wie Internet of Things (IoT) und Machine-to-Machine-Kommunikation (M2M) treibt die Digitalisierung und Automatisierung in der Produktion weiter voran. Die dabei generierten Datenmengen ermöglichen moderne für Predictive Maintenance-Ansätze. Durch die Analyse von Maschinen- und Produktionsdaten können Maschinen proaktiv gewartet werden, um dadurch die Qualitätsstandards und Produktivität zu erhöhen. Da Kugelgewindetriebe  in Industriemaschinen häufig verbaut werden, ist deren Zustandsüberwachung besonders interessant. Durch den hohen Verschleiß verändern sich Kugelgewindetriebe über ihrer Lebensdauer stark. Zwischen den Betriebszuständen ist deshalb eine starke Verschiebung in der Datenverteilung zu beobachten. Dies macht die Fehlerdiagnose herausfordernd. Deep-Learning-basierte Domain-Adaption Ansätze, die in der Computer-Vision-Community verbreitet sind, sind daher vielversprechend. In dieser Arbeit wird die Anwendbarkeit solcher Ansätze für das gegebene Predictive Maintenance Probelem untersucht. 


\end{otherlanguage}


% Undo the name switch
\makeatletter
\ifthenelse{\pdf@strcmp{\languagename}{english}=0}
{\renewcommand{\abstractname}{Abstract}}
{\renewcommand{\abstractname}{Kurzfassung}}
\makeatother